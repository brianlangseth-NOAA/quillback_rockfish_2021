\input{input_accessability.tex}
\documentclass[11pt,
  english,
  a4paper,
]{article}
\usepackage{sa4ss}
\usepackage{amsmath,amssymb,array}
\usepackage{booktabs}

% From tagged-template.latex
\usepackage{lmodern}
\usepackage{ifxetex,ifluatex}
\ifnum 0\ifxetex 1\fi\ifluatex 1\fi=0 % if pdftex
  \usepackage[T1]{fontenc}
  \usepackage[utf8]{inputenc}
  \usepackage{textcomp} % provide euro and other symbols
\else % if luatex or xetex
  \usepackage{unicode-math}
  \defaultfontfeatures{Scale=MatchLowercase}
  \defaultfontfeatures[\rmfamily]{Ligatures=TeX,Scale=1}
\fi

% Use upquote if available, for straight quotes in verbatim environments
\IfFileExists{upquote.sty}{\usepackage{upquote}}{}
\IfFileExists{microtype.sty}{% use microtype if available
  \usepackage[]{microtype}
  \UseMicrotypeSet[protrusion]{basicmath} % disable protrusion for tt fonts
}{}
\makeatletter
\@ifundefined{KOMAClassName}{% if non-KOMA class
  \IfFileExists{parskip.sty}{%
    \usepackage{parskip}
  }{% else
    \setlength{\parindent}{0pt}
    \setlength{\parskip}{6pt plus 2pt minus 1pt}}
}{% if KOMA class
  \KOMAoptions{parskip=half}}
\makeatother
\usepackage{xcolor}
\IfFileExists{xurl.sty}{\usepackage{xurl}}{} % add URL line breaks if available
\hypersetup{
  pdftitle={Rebuilding analysis for quillback rockfish (Sebastes maliger) in U.S. waters off the coast of California based on the 2021 stock assessment},
  pdflang={en},
  hidelinks,
  pdfcreator={LaTeX via pandoc}}
\urlstyle{same} % disable monospaced font for URLs
\usepackage{longtable}
% Correct order of tables after \paragraph or \subparagraph
\usepackage{etoolbox}
\makeatletter
\patchcmd\longtable{\par}{\if@noskipsec\mbox{}\fi\par}{}{}
\makeatother
% Allow footnotes in longtable head/foot
\IfFileExists{footnotehyper.sty}{\usepackage{footnotehyper}}{\usepackage{footnote}}
\makesavenoteenv{longtable}
\usepackage{graphicx}
\makeatletter
\def\maxwidth{\ifdim\Gin@nat@width>\linewidth\linewidth\else\Gin@nat@width\fi}
\def\maxheight{\ifdim\Gin@nat@height>\textheight\textheight\else\Gin@nat@height\fi}
\makeatother
% Scale images if necessary, so that they will not overflow the page
% margins by default, and it is still possible to overwrite the defaults
% using explicit options in \includegraphics[width, height, ...]{}
\setkeys{Gin}{width=\maxwidth,height=\maxheight,keepaspectratio}
% Set default figure placement to htbp
\makeatletter
\def\fps@figure{htbp}
\makeatother
\setlength{\emergencystretch}{3em} % prevent overfull lines
\providecommand{\tightlist}{%
  \setlength{\itemsep}{0pt}\setlength{\parskip}{0pt}}
\setcounter{secnumdepth}{5}
\ifxetex
  % Load polyglossia as late as possible: uses bidi with RTL langages (e.g. Hebrew, Arabic)
  \usepackage{polyglossia}
  \setmainlanguage[]{english}
\else
  \usepackage[shorthands=off,main=english]{babel}
\fi

\providecommand{\tightlist}{%
  \setlength{\itemsep}{0pt}\setlength{\parskip}{0pt}}


\date{}
\newcommand{\trTitle}{Rebuilding analysis for quillback rockfish (\emph{Sebastes maliger}) in U.S. waters off the coast of California based on the 2021 stock assessment}
\newcommand{\trYear}{2021}
\newcommand{\trMonth}{September}
\newcommand{\trAuthsLong}{true}
\newcommand{\trAuthsBack}{Langseth, B.J., C.R. Wetzel}
\newcommand{\trCitation}{
\begin{hangparas}{1em}{1}
\trAuthsBack{}. \trYear{}. \trTitle{}. Pacific Fisheries Management Council, Portland, Oregon. \pageref{LastPage}{}\,p.
\end{hangparas}}

\AtBeginDocument{\tagstructbegin{tag=Document}}
\AtEndDocument{\tagstructend}
\pretocmd{\maketitle}{\tagstructbegin{tag=H1}\tagmcbegin{tag=H1}}{}{}
\apptocmd{\maketitle}{\tagmcend\tagstructend}{}{}

\begin{document}

%%%%% Frontmatter %%%%%

% Footnote symbols in front matter
\renewcommand*{\thefootnote}{\fnsymbol{footnote}}

\small
\thispagestyle{empty}
\pagenumbering{roman}
\noindent
\begin{center}
\title{Rebuilding analysis for quillback rockfish (\emph{Sebastes maliger}) in U.S. waters off the coast of California based on the 2021 stock assessment}
% \textnormal{\MakeTextUppercase{\trTitle{}}}
\vspace{1.5cm}
{\Large\textbf\newline{Rebuilding analysis for quillback rockfish (\emph{Sebastes maliger}) in U.S. waters off the coast of California based on the 2021 stock assessment}}
\vfill
by\\
Brian J. Langseth\textsuperscript{1}\\
Chantel R. Wetzel\textsuperscript{1}\vfill
\textsuperscript{1}Northwest Fisheries Science Center, U.S. Department of Commerce, National Oceanic and Atmospheric Administration, National Marine Fisheries Service, 2725 Montlake Boulevard East, Seattle, Washington 98112\vfill
\trMonth{} \trYear{}
\end{center}
\clearpage

% Fourth page: Colophon
\thispagestyle{empty}
\vspace*{\fill}
\begin{center}
\copyright{} Pacific Fisheries Management Council, \trYear{}\\
\end{center}
\par
\bigskip
\noindent
Correct citation for this publication:
\bigskip
\par
\trCitation{}
\clearpage

% Add TOC to pdf bookmarks (clickable pdf)
\pdfbookmark[1]{\contentsname}{toc}

% Table of contents page, lists of figures and tables
\tableofcontents\clearpage
%\listoffigures \listoftables \clearpage
\label{TRlastRoman}
\clearpage

% Table of contents
\newpage
\thispagestyle{empty} % to remove page number

% Settings for the main document
\pagenumbering{arabic}  % Regular page numbers
\pagestyle{plain}  % No page number on first page of main document, use 'empty'
\renewcommand*{\thefootnote}{\arabic{footnote}}  % Back to numeric footnotes
\setcounter{footnote}{0}  % And start at 1
\renewcommand{\headrulewidth}{0.5pt}
\renewcommand{\footrulewidth}{0.5pt}
%\pagestyle{fancy}\fancyhead[c]{Draft: Do not cite or circulate}

\newcommand{\lt}{\ensuremath <}
\newcommand{\gt}{\ensuremath >}

%Define cslreferences environment, required by pandoc 2.8
%https://github.com/rstudio/rmarkdown/issues/1649

\pagebreak
\pagenumbering{roman}
\setcounter{page}{1}

\renewcommand{\thetable}{\roman{table}}
\renewcommand{\thefigure}{\roman{figure}}

\setlength\parskip{0.5em plus 0.1em minus 0.2em}

\tagstructbegin{tag=H1}\tagmcbegin{tag=H1}

\hypertarget{summary}{%
\section*{Summary}\label{summary}}
\addcontentsline{toc}{section}{Summary}

\leavevmode\tagmcend\tagstructend

\tagstructbegin{tag=P}\tagmcbegin{tag=P}

This rebuilding analysis is for the stock of quillback rockfish (\emph{Sebastes maliger}) in waters off California. The analysis is based on the 2021 stock assessment. The 2021 assessment model estimated the quillback rockfish population to be at 14\% of the unexploited equilibrium spawning output at the beginning of 2021. This rebuilding analysis compares the results of applying a suite of potential management actions to the stock for 2023 and beyond.

\leavevmode\tagmcend\tagstructend\par

\tagstructbegin{tag=P}\tagmcbegin{tag=P}

The results of the analysis show that the value for {\tagstructbegin{tag=Formula}\tagmcbegin{tag=Formula}\(\text{T}_\text{MIN}\)\leavevmode\tagmcend\tagstructend}, the median year for rebuilding to the target level in the absence of fishing since the year of declaration (2023), is 2040. The estimated generation time for quillback rockfish was estimated to be 27 years. In conjunction with {\tagstructbegin{tag=Formula}\tagmcbegin{tag=Formula}\(\text{T}_\text{MIN}\)\leavevmode\tagmcend\tagstructend} and the mean generation time, {\tagstructbegin{tag=Formula}\tagmcbegin{tag=Formula}\(\text{T}_\text{MAX}\)\leavevmode\tagmcend\tagstructend} was estimated to be 2067. The SPR = 0.537 harvest rate generates a 50\% probability of recovery by {\tagstructbegin{tag=Formula}\tagmcbegin{tag=Formula}\(\text{T}_\text{MID}\)\leavevmode\tagmcend\tagstructend} where {\tagstructbegin{tag=Formula}\tagmcbegin{tag=Formula}\(\text{T}_\text{MID}\)\leavevmode\tagmcend\tagstructend} was set equal to 2054, an intermediate year between {\tagstructbegin{tag=Formula}\tagmcbegin{tag=Formula}\(\text{T}_\text{MIN}\)\leavevmode\tagmcend\tagstructend} and {\tagstructbegin{tag=Formula}\tagmcbegin{tag=Formula}\(\text{T}_\text{MAX}\)\leavevmode\tagmcend\tagstructend}.

\leavevmode\tagmcend\tagstructend\par

\pagebreak
\setlength{\parskip}{5mm plus1mm minus1mm}
\pagenumbering{arabic}
\setcounter{page}{1}
\renewcommand{\thefigure}{\arabic{figure}}
\renewcommand{\thetable}{\arabic{table}}
\setcounter{table}{0}
\setcounter{figure}{0}

\setlength\parskip{0.2em plus 0.1em minus 0.2em}

\tagstructbegin{tag=H1}\tagmcbegin{tag=H1}

\hypertarget{introduction}{%
\section{Introduction}\label{introduction}}

\leavevmode\tagmcend\tagstructend

\tagstructbegin{tag=P}\tagmcbegin{tag=P}

The 2021 assessment of quillback rockfish (\emph{Sebastes maliger}) in California waters documented that the population of quillback rockfish was below the Minimum Stock Size Threshold (MSST), which is 25\% of unfished spawning output for rockfish stocks, in 2021. The population had declined below MSST starting in 1992, reached it lowest values in the mid-1990's but increased to near the MSST in the 2000s and early 2010s, and had declined in recent years {\tagstructbegin{tag=Reference}\tagmcbegin{tag=Reference}(Langseth et al. 2021)\leavevmode\tagmcend\tagstructend}. The stock was declared overfished for 2023 in 2021. Given the assumed productivity of the stock combined with the longevity of quillback rockfish a range of alternative rebuilding approaches were examined with rebuilding ranging from 2040 to 2067 based on various SPR harvest rates from 0.54 to 1

\leavevmode\tagmcend\tagstructend\par

\tagstructbegin{tag=H1}\tagmcbegin{tag=H1}

\hypertarget{overview-of-the-2021-stock-assessment}{%
\section{Overview of the 2021 stock assessment}\label{overview-of-the-2021-stock-assessment}}

\leavevmode\tagmcend\tagstructend

\tagstructbegin{tag=P}\tagmcbegin{tag=P}

The 2021 assessment of quillback rockfish assessed the stock as three separate sub-stocks along the U.S. west coast: Washington, Oregon, and California. These were the first assessments of quillback rockfish conducted within Stock Synthesis {\tagstructbegin{tag=Reference}\tagmcbegin{tag=Reference}(Methot and Wetzel 2013)\leavevmode\tagmcend\tagstructend} that used catch and length composition data to inform model estimates around stock size and status. The previous assessment of quillback rockfish conducted in 2010 was a coastwide assessment modeled using Depletion-Based Stock Reduction Analysis (DB-SRA) to provide estimates of coastwide OFLs using catch data and biological information {\tagstructbegin{tag=Reference}\tagmcbegin{tag=Reference}(Dick and MacCall 2010)\leavevmode\tagmcend\tagstructend}. DB-SRA is a catch-only model and does not assess overfished status; the 2010 assessment assumed that current depletion was distributed around the management target of 40\%. The 2010 assessment found there was a 52\% chance that quillback rockfish was experiencing overfishing, as recent coastwide catch of quillback rockfish slightly exceeded the median coastwide OFL estimate at the time. Recent catches of quillback rockfish for the current assessment also exceed the ACL contributions for the species in all modeled areas.

\leavevmode\tagmcend\tagstructend\par

\tagstructbegin{tag=P}\tagmcbegin{tag=P}

Estimates of depletion in 2021 for the sub-stocks off Washington and Oregon were above the MSST threshold, but the estimate of depletion for the sub-stock off California was 0.14. The median and 95\% confidence intervals of select derived quantities for the California quillback rockfish base model are provided in (Table \ref{tab:model-summary}).

\leavevmode\tagmcend\tagstructend\par

\tagstructbegin{tag=P}\tagmcbegin{tag=P}

California quillback rockfish was assessed using a one-sex model with coastwide life history parameters combined across sexes. Life history parameters were estimated externally and then fixed within the model. Natural mortality and steepness were both fixed at the median and mean of the priors, respectively. Annual recruitment deviations were estimated within the base model. The model for Quillback rockfish in California waters included two fishing fleets, a commercial and a recreational fleet. The majority of the removals and length composition data arose from the recreational fleet. Recreational removals peaked in the late 1970s and early 1980s, with two years of large catches in 1984 and 1993. Removals declined sharply in 1994, but increased to levels similar to the late 1970s and early 1980s in the mid 2000s and again in recent years. Commercial removals peaked in the mid to late 1990s, with one year of exceptionally large catches in 1991. Removals declined through the mid 2010s, but increased in recent years. Selectivity for the commercial and recreational fleets was specified to be asymptotic. The assessment model decision table explored uncertainty around stock size and status using lower and higher natural mortality values relative to the base model.

\leavevmode\tagmcend\tagstructend\par

\tagstructbegin{tag=P}\tagmcbegin{tag=P}

Sensitivities to modeling choices and parameter values were explored and showed general support for the base model estimates of stock status and depletion. Sensitivities to the von Bertalanffy growth coefficient ({\tagstructbegin{tag=Formula}\tagmcbegin{tag=Formula}\(k\)\leavevmode\tagmcend\tagstructend}) and natural mortality ({\tagstructbegin{tag=Formula}\tagmcbegin{tag=Formula}\(M\)\leavevmode\tagmcend\tagstructend}) showed that model estimates of depletion were sensitive to these parameter choices.

\leavevmode\tagmcend\tagstructend\par

\tagstructbegin{tag=P}\tagmcbegin{tag=P}

TO DO: Need median and 95\% intervals for desired quantities

\leavevmode\tagmcend\tagstructend\par

\tagstructbegin{tag=H1}\tagmcbegin{tag=H1}

\hypertarget{management-performance-under-rebuilding}{%
\section{Management performance under rebuilding}\label{management-performance-under-rebuilding}}

\leavevmode\tagmcend\tagstructend

\tagstructbegin{tag=P}\tagmcbegin{tag=P}

This is the first rebuilding plan for quillback rockfish in waters off the coast of California.

\leavevmode\tagmcend\tagstructend\par

\tagstructbegin{tag=H1}\tagmcbegin{tag=H1}

\hypertarget{rebuilding-calculations}{%
\section{Rebuilding calculations}\label{rebuilding-calculations}}

\leavevmode\tagmcend\tagstructend

\tagstructbegin{tag=P}\tagmcbegin{tag=P}

This rebuilding analysis was conducted in September, 2021 using software developed by A. Punt (version 3.12h, August 2021). Input files for the analysis are provided in {\tagstructbegin{tag=Link}\tagmcbegin{tag=Link}\protect\hyperlink{append_a}{Appendix A}\leavevmode\tagmcend\tagstructend}. The steps followed were:

\leavevmode\tagmcend\tagstructend\par

\begin{enumerate}
    \item Define how equilibrium spawning output ($\text{SB}_0$) will be calculated. 
    \item Define how future recruitment will be generated.
    \item Define the biological information on which future projections will be based.
    \item Define the fishery selectivity and allocation to be applied during rebuilding. 
    \item Decide how to include uncertainty in input parameters from the stock assessment in the rebuilding analysis. 
    \item Calculate rebuilding reference points from the most current assessment results 
    \begin{itemize}
        \item Calculate the projected year in which the stock would rebuild with a 50$\%$ probability if all future fishing mortality was eliminated ($\text{T}_\text{F}$=0).
        \item  Calculate the projected year for a 50$\%$ probability of rebuilding from the year in which the stock was first declared overfished ($\text{T}_\text{MIN}$). 
        \item Calculate the mean generation time. 
        \item Calculate the maximum allowable rebuilding time ($\text{T}_\text{MAX}$). 
    \end{itemize}
    \item Identification and analysis of alternative harvest strategies for rebuilding. 
\end{enumerate}

\tagstructbegin{tag=P}\tagmcbegin{tag=P}

TO DO: I dont think the ``calculate rebuilding reference points'' section is needed. Not sure why I added it. Its for the results and not included in the TOR description.

\leavevmode\tagmcend\tagstructend\par

\tagstructbegin{tag=H2}\tagmcbegin{tag=H2}

\hypertarget{definition-of-equilibrium-spawning-output}{%
\subsection{Definition of Equilibrium Spawning Output}\label{definition-of-equilibrium-spawning-output}}

\leavevmode\tagmcend\tagstructend

\tagstructbegin{tag=P}\tagmcbegin{tag=P}

The equilibrium spawning output ({\tagstructbegin{tag=Formula}\tagmcbegin{tag=Formula}\(\text{SB}_0\)\leavevmode\tagmcend\tagstructend}) used in this rebuilding analysis is calculated via the stock-recruitment, growth, maturity, and fecundity relationships from the 2021 assessment in order to be consistent with assessment model results. Equilibrium spawning output was estimated to be 55.08 millions of eggs in the assessment model, which dictates a rebuilding relative spawning output target ({\tagstructbegin{tag=Formula}\tagmcbegin{tag=Formula}\(\text{SB}_{40\%}\)\leavevmode\tagmcend\tagstructend}) of 22.035 millions of eggs (Table \ref{tab:ref-points}). Estimates of spawning output presented in this report are female spawning output only.

\leavevmode\tagmcend\tagstructend\par

\tagstructbegin{tag=H2}\tagmcbegin{tag=H2}

\hypertarget{generation-of-future-recruitment}{%
\subsection{Generation of future recruitment}\label{generation-of-future-recruitment}}

\leavevmode\tagmcend\tagstructend

\tagstructbegin{tag=P}\tagmcbegin{tag=P}

The estimated parameters of the stock recruitment relationship including the unexploited equilibrium recruitment (natural log of {\tagstructbegin{tag=Formula}\tagmcbegin{tag=Formula}\(R_0\)\leavevmode\tagmcend\tagstructend} = 3.168), steepness ({\tagstructbegin{tag=Formula}\tagmcbegin{tag=Formula}\(h\)\leavevmode\tagmcend\tagstructend} = 0.72), and degree of recruitment variability ({\tagstructbegin{tag=Formula}\tagmcbegin{tag=Formula}\(\sigma_R = 0.60\)\leavevmode\tagmcend\tagstructend}) from the 2021 assessment were used to generate future recruitments in the rebuilding analysis.

\leavevmode\tagmcend\tagstructend\par

\tagstructbegin{tag=H2}\tagmcbegin{tag=H2}

\hypertarget{population-biology}{%
\subsection{Population biology}\label{population-biology}}

\leavevmode\tagmcend\tagstructend

\tagstructbegin{tag=P}\tagmcbegin{tag=P}

The biological parameters used for the rebuilding analysis were based on the values from the 2021 assessment. Biological parameters in the assessment were aggregated across sex and constant across time. The rebuilding analysis was based on a single sex model. Biological values used as input into the rebuilding analysis are provided in (Table \ref{tab:biology}).

\leavevmode\tagmcend\tagstructend\par

\tagstructbegin{tag=H2}\tagmcbegin{tag=H2}

\hypertarget{fishery-selectivity-and-removal-allocations}{%
\subsection{Fishery selectivity, and removal allocations}\label{fishery-selectivity-and-removal-allocations}}

\leavevmode\tagmcend\tagstructend

\tagstructbegin{tag=P}\tagmcbegin{tag=P}

The selectivity used in the rebuilding analysis was obtained from the 2021 assessment. Selectivity in the assessment model was constant across time for each fishing fleet. The relative allocation of catch among fleets in the rebuilding analysis was informed using the relative fishing mortality from the assessment averaged over recent years (2017-2019). This choice provides some consistency between recent model results and forecasts from the rebuilding analysis, accounting for the unique dynamics in 2020 caused by the Covid-19 pandemic.

\leavevmode\tagmcend\tagstructend\par

\tagstructbegin{tag=H2}\tagmcbegin{tag=H2}

\hypertarget{inclusion-of-uncertainty}{%
\subsection{Inclusion of uncertainty}\label{inclusion-of-uncertainty}}

\leavevmode\tagmcend\tagstructend

\tagstructbegin{tag=P}\tagmcbegin{tag=P}

Model and parameter uncertainty is included in the rebuilding analysis via 1,000 random simulations of stochastic future recruitment strengths and integration over alternative low ({\tagstructbegin{tag=Formula}\tagmcbegin{tag=Formula}\(M\)\leavevmode\tagmcend\tagstructend} = 0.0464) and high ({\tagstructbegin{tag=Formula}\tagmcbegin{tag=Formula}\(M\)\leavevmode\tagmcend\tagstructend} = 0.0744) states of nature for values of natural mortality ({\tagstructbegin{tag=Formula}\tagmcbegin{tag=Formula}\(M\)\leavevmode\tagmcend\tagstructend}). Other potential states of nature were explored, including for low and high ln({\tagstructbegin{tag=Formula}\tagmcbegin{tag=Formula}\(R0\)\leavevmode\tagmcend\tagstructend}) and high and low {\tagstructbegin{tag=Formula}\tagmcbegin{tag=Formula}\(L_{\infty}\)\leavevmode\tagmcend\tagstructend}, but alternative values of natural mortality encapsulated a slightly wider range of depletion estimates compared to the other states of natures. The base model was given 50\% of the weight and each alternative natural mortality state of nature was given 25\% of the weight.

\leavevmode\tagmcend\tagstructend\par

\tagstructbegin{tag=H2}\tagmcbegin{tag=H2}

\hypertarget{alternate-rebuilding-strategies-analyzed}{%
\subsection{Alternate rebuilding strategies analyzed}\label{alternate-rebuilding-strategies-analyzed}}

\leavevmode\tagmcend\tagstructend

\tagstructbegin{tag=P}\tagmcbegin{tag=P}

Assuming that a constant rate of harvest will be applied throughout a rebuilding period, the basis for rebuilding alternatives can be divided into two approaches: 1) strategies based on selection of a constant harvest rate (SPR rate), or 2) strategies based on selection of a {\tagstructbegin{tag=Formula}\tagmcbegin{tag=Formula}\(\text{T}_\text{TARGET}\)\leavevmode\tagmcend\tagstructend} (year for 50\% probability of recovery). This rebuilding analysis presents the following alternate strategies, which are a combination of those specified in the rebuilding Terms of Reference (TOR) and additional strategies. The additional strategies based on the selection of a SPR harvest rate or rebuilding by a selected target year:

\leavevmode\tagmcend\tagstructend\par

\begin{enumerate}
    \item Eliminate all harvest, F = 0, beginning in the next management cycle, 2023, the same as setting a constant SPR harvest rate of 1.0.
    \item Apply the harvest rate that would generate the ACL specified for the current year (i.e., the latest year specified in regulations).
    \item Apply a range of SPR values: 
    \begin{enumerate}
     \item SPR = 0.50,
     \item SPR = 0.60,
     \item SPR = 0.70,
     \item SPR = 0.80, and
     \item SPR = 0.90. 
    \end{enumerate}
    \item Apply SPR harvest rates that are estimated to lead to a 50$\%$ probability of recovery by alternative target years: 
    \begin{enumerate}
     \item by $\text{T}_\text{MAX}$ from the current cycle, and 
     \item by $\text{T}_\text{MID}$ from the current cycle, which is the year midway between $\text{T}_\text{MIN}$ and $\text{T}_\text{MAX}$.
    \end{enumerate}
    \item Apply the default harvest policy based on the 40:10 harvest control rule.
    \item Apply the ABC harvest rate with time-varying category 2 $\sigma$ and $P^*$ = 0.45.
\end{enumerate}

\tagstructbegin{tag=P}\tagmcbegin{tag=P}

No current rebuilding plan exists for quillback rockfish in California waters so the following alternatives as specified in the rebuilding TOR could not be done:

\leavevmode\tagmcend\tagstructend\par

\begin{itemize}
 \item Apply the spawning potential ratio or relevant harvest control run in the current rebuilding plan. 
 \item Apply the harvest rate that is estimated to lead to a 50% probability of recovery by the current $\text{T}_\text{TARGET}$.
 \item Apply the harvest rate that is estimated to lead to a 50% probability of recovery by the TMAX from the previous cycle.
\end{itemize}

\tagstructbegin{tag=P}\tagmcbegin{tag=P}

TO DO: Add a run that generates the currently specified ACL. But there is no ACL, its an ACL contribution, so do I need to do this? Not sure how to do this.

\leavevmode\tagmcend\tagstructend\par

\tagstructbegin{tag=P}\tagmcbegin{tag=P}

TO DO: Add a run with Tmax. Not sure how to do this.

\leavevmode\tagmcend\tagstructend\par

\tagstructbegin{tag=P}\tagmcbegin{tag=P}

All of the above rebuilding strategies were conducted assuming removals of 13.5 mt in 2021 and 2022 as recommended by the Groundfish Management Team (GMT).

\leavevmode\tagmcend\tagstructend\par

\tagstructbegin{tag=P}\tagmcbegin{tag=P}

The Pacific Fishery Management Council's (Council) Scientific and Statistical Committee (SSC) requested at the September Council meeting that an additional rebuilding sensitivity be done examining the impact of blocking both recreational and commercial selectivity at 1916-2000 and 2001-2020, assuming asymptotic selectivity for the early blocks and allowing estimation of dome-shaped selectivity for the recent blocks. This sensitivity was conducted without integration over alternative states of nature for natural mortality, given its treatment as a sensitivity. The biological parameters on which the projections are based for this sensitivity are provided in {\tagstructbegin{tag=Link}\tagmcbegin{tag=Link}\protect\hyperlink{append_b}{Appendix B}\leavevmode\tagmcend\tagstructend} (Table \ref{tab:sens-biology}).

\leavevmode\tagmcend\tagstructend\par

\tagstructbegin{tag=P}\tagmcbegin{tag=P}

TO DO: Add this biology table for the sensitivity

\leavevmode\tagmcend\tagstructend\par

\tagstructbegin{tag=H1}\tagmcbegin{tag=H1}

\hypertarget{results}{%
\section{Results}\label{results}}

\leavevmode\tagmcend\tagstructend

\tagstructbegin{tag=H2}\tagmcbegin{tag=H2}

\hypertarget{rebuilding-reference-points}{%
\subsection{Rebuilding reference points}\label{rebuilding-reference-points}}

\leavevmode\tagmcend\tagstructend

\tagstructbegin{tag=P}\tagmcbegin{tag=P}

Reference points calculated based on this rebuilding analysis are given in Table \ref{tab:ref-points}. The minimum time required for rebuilding, {\tagstructbegin{tag=Formula}\tagmcbegin{tag=Formula}\(\text{T}_\text{MIN}\)\leavevmode\tagmcend\tagstructend}, with no fishing (F=0) starting in 2023 was estimated to be 19 years, corresponding to the stock being rebuilt by 2040, assuming the default removals for 2021 and 2022. The mean generation time was estimated to be 27 years. The maximum time allowed for rebuilding, {\tagstructbegin{tag=Formula}\tagmcbegin{tag=Formula}\(\text{T}_\text{MAX}\)\leavevmode\tagmcend\tagstructend}, is defined as the {\tagstructbegin{tag=Formula}\tagmcbegin{tag=Formula}\(\text{T}_\text{MIN}\)\leavevmode\tagmcend\tagstructend} plus the mean generation time for stocks that require more than 10 years to rebuild. Quillback rockfish was unable to rebuild within 10 years so the estimated {\tagstructbegin{tag=Formula}\tagmcbegin{tag=Formula}\(\text{T}_\text{MAX}\)\leavevmode\tagmcend\tagstructend} was 2067. {\tagstructbegin{tag=Formula}\tagmcbegin{tag=Formula}\(\text{P}_\text{MAX}\)\leavevmode\tagmcend\tagstructend}, {\tagstructbegin{tag=Formula}\tagmcbegin{tag=Formula}\(\text{T}_\text{TARGET}\)\leavevmode\tagmcend\tagstructend}, and {\tagstructbegin{tag=Formula}\tagmcbegin{tag=Formula}\(\text{SPR}_\text{TARGET}\)\leavevmode\tagmcend\tagstructend} are not specified since this is the first rebuilding plan for quillback rockfish and these values have not been set via the Council's process.

\leavevmode\tagmcend\tagstructend\par

\tagstructbegin{tag=P}\tagmcbegin{tag=P}

TO DO: Confirm Tmin value (17 or 19) and Tmin year (2040 or 2042). Tmax assumes Tminyear is 2040 so ensure values are accurate here and in tables.

\leavevmode\tagmcend\tagstructend\par

\tagstructbegin{tag=H2}\tagmcbegin{tag=H2}

\hypertarget{alternative-harvest-policy-projections}{%
\subsection{Alternative harvest policy projections}\label{alternative-harvest-policy-projections}}

\leavevmode\tagmcend\tagstructend

\tagstructbegin{tag=P}\tagmcbegin{tag=P}

Summary results from the rebuilding alternatives are presented in Table \ref{tab:ref-options}. Detailed results are presented in Tables \ref{tab:prob-mat}, \ref{tab:acl-mat}, \ref{tab:ofl-mat}, \ref{tab:rel-ssb-mat} and Figures \ref{fig:prob-fig}, \ref{fig:acl-fig}, \ref{fig:rel-ssb-fig}, \ref{fig:ssb-fig}.

\leavevmode\tagmcend\tagstructend\par

\tagstructbegin{tag=P}\tagmcbegin{tag=P}

The target rebuilding year based on the range of pre-specified SPR values between 0.50 - 0.90 ranged from 2040 - 2064 (Table \ref{tab:reb-options}). The probability of rebuilding by year steadily increased across the alternative SPR values with full rebuilding (reaching a probability of greater than 50\%) by 2064 when the lowest SPR of 0.50 was applied and catches were not constrained to the ABC (Table \ref{tab:prob-mat} and Figure \ref{fig:prob-fig}). The recommended removals in 2023, the first year of rebuilding, were low ranging between 0.25 - 1.79 mt across alternative SPR values, and 0.04 - 1.79 across alternative strategies allowing some amount of catch (Table \ref{tab:acl-mat}), with the recommended removals slowly increasing by year during the rebuilding period (Figure \ref{fig:acl-fig}). The estimated Overfishing Limits (OFLs) for each rebuilding alternative is given in Table \ref{tab:ofl-mat}. The change in spawning output by year relative to the spawning output target, 40\% of unfished, under each of the alternatives are shown by year in Table \ref{tab:rel-ssb-mat} and Figure \ref{fig:rel-ssb-fig}).

\leavevmode\tagmcend\tagstructend\par

\tagstructbegin{tag=P}\tagmcbegin{tag=P}

TO DO: Automate presentation of the years in the first sentence

\leavevmode\tagmcend\tagstructend\par

\tagstructbegin{tag=P}\tagmcbegin{tag=P}

TO DO: Ensure min max catch values reflect the ABC caps

\leavevmode\tagmcend\tagstructend\par

\tagstructbegin{tag=P}\tagmcbegin{tag=P}

TO DO: Set number of table rows to be capped at Tmax

\leavevmode\tagmcend\tagstructend\par

\tagstructbegin{tag=P}\tagmcbegin{tag=P}

TO DO: provide an asterisk where ABC caps were applied

\leavevmode\tagmcend\tagstructend\par

\tagstructbegin{tag=P}\tagmcbegin{tag=P}

TO DO: Set probabilities in tables to three digits

\leavevmode\tagmcend\tagstructend\par

\tagstructbegin{tag=P}\tagmcbegin{tag=P}

TO DO?: This seems short, is more results needed?

\leavevmode\tagmcend\tagstructend\par

\tagstructbegin{tag=H2}\tagmcbegin{tag=H2}

\hypertarget{results-for-sensitivity-on-selectivity-blocks-and-selectivity-form}{%
\subsection{Results for sensitivity on selectivity blocks and selectivity form}\label{results-for-sensitivity-on-selectivity-blocks-and-selectivity-form}}

\leavevmode\tagmcend\tagstructend

\tagstructbegin{tag=P}\tagmcbegin{tag=P}

Reference points and summary results for the sensitivity rebuilding analysis that was based on a model with selectivity blocks for both the recreational and commercial fleets applied in 2001 with dome-shaped selectivity in the recent blocks are shown in Tables \ref{tab:ref-points-block} and \ref{tab:ref-options-block}. Comparisons between stock synthesis output from the adopted base model and the sensitivity run are provided in \protect\hypertarget{append_c}{}{Appendix C}. The sensitivity had slightly higher estimates of unfished spawning output (Figure \ref{fig:ssb-sens}) and similar depletion estimates (Figure \ref{fig:depl-sens}) compared to the adopted base model. Selectivity in the recent block was domed at large sizes for both the recreational and commercial fleets, and right shifted in recent years for the commercial fleet (Figure \ref{fig:selex-sens}) so the numbers at age entered into the sensitivity rebuilding analysis were different from the numbers at age entered into the base rebuilding analysis, contributing to differences in results.

\leavevmode\tagmcend\tagstructend\par

\tagstructbegin{tag=P}\tagmcbegin{tag=P}

However, we do not consider the sensitivity model to be an improvement over the base model. The shifts in the selectivity curve for the commercial fleet fit the lower observed mean lengths in the 1990s and the higher observed mean lengths starting around 2010 better than the base model did (Figures \ref{fig:MeanLengthFit-base} and \ref{fig:MeanLengthFit-sens}), leading to an improvement in AIC. However the mode and right side of the estimated aggregate length distribution from the commercial fleet visually fit the data worse than the base model did (Figures \ref{fig:aggLenFit-sens} and \ref{fig:aggLenFit-sens}). Combined with the similarities in population scale and status, the base model is preferred due to parsimony. Improved understanding of where the fleets are fishing and sizes of quillback rockfish available by depth would better inform the selectivity curve for future assessments of this species.

\leavevmode\tagmcend\tagstructend\par

\tagstructbegin{tag=P}\tagmcbegin{tag=P}

QUESTION: Is there more fit material needed. The amount of it seems extraneous to the rebuilding analysis.

\leavevmode\tagmcend\tagstructend\par

\tagstructbegin{tag=H1}\tagmcbegin{tag=H1}

\hypertarget{acknowledgements}{%
\section{Acknowledgements}\label{acknowledgements}}

\leavevmode\tagmcend\tagstructend

\tagstructbegin{tag=P}\tagmcbegin{tag=P}

Andre Punt (University of Washington) quickly updated the rebuilder program to apply time-varying {\tagstructbegin{tag=Formula}\tagmcbegin{tag=Formula}\(\sigma\)\leavevmode\tagmcend\tagstructend} for the Acceptable Biological Catch scenarios. Andre Punt and Owen Hamel (NOAA) provided assistant in understanding and applying the rebuilder program. Owen Hamel provided comments on early drafts of the analysis.

\leavevmode\tagmcend\tagstructend\par

\clearpage

\tagstructbegin{tag=H1}\tagmcbegin{tag=H1}

\hypertarget{references}{%
\section{References}\label{references}}

\leavevmode\tagmcend\tagstructend

\tagstructbegin{tag=BibEntry}\tagmcbegin{tag=BibEntry}

\hypertarget{refs}{}
\leavevmode\hypertarget{ref-DickandMacCall_dbsra_2010}{}%
Dick, E. J., and Alec D. MacCall. 2010. ``Estimates of Sustainable Yield for 50 Data-Poor Stocks in the Pacfici Coast Groundfish Fishery Management Plot.'' NOAA-TM-NMFS-SWFSC-460. NOAA Tech. Memo.

\leavevmode\hypertarget{ref-Langseth_status_2021}{}%
Langseth, Brian J., Chantel R. Wetzel, Jason M. Cope, and John E. Budrick. 2021. ``Status of Quillback Rockfish (\emph{Sebastes Maliger}) in U.S. Waters Off the Coast of California in 2021 Using Catch and Length Data.'' 7700 Ambassador Place NE, Suite 200, Portland, OR 97220: Pacific Fishery Management Council.

\leavevmode\hypertarget{ref-methot_stock_2013}{}%
Methot, R. D., and C. R. Wetzel. 2013. ``Stock Synthesis: A Biological and Statistical Framework for Fish Stock Assessment and Fishery Management.'' \emph{Fisheries Research} 142 (May): 86--99. \url{https://doi.org/10.1016/j.fishres.2012.10.012}.

\leavevmode\tagmcend\tagstructend

\clearpage

\tagstructbegin{tag=H1}\tagmcbegin{tag=H1}

\hypertarget{tables}{%
\section{Tables}\label{tables}}

\leavevmode\tagmcend\tagstructend

\tagstructbegin{tag=P}\tagmcbegin{tag=P}

TO DO: Ensure spawning outputs reflect female spawning output

\leavevmode\tagmcend\tagstructend\par

\tagstructbegin{tag=P}\tagmcbegin{tag=P}

TO DO: Add sensitivity rebuilder results

\leavevmode\tagmcend\tagstructend\par

\tagstructbegin{tag=H2}\tagmcbegin{tag=H2}

\hypertarget{rebuilding-reference-points-1}{%
\subsection{Rebuilding reference points}\label{rebuilding-reference-points-1}}

\leavevmode\tagmcend\tagstructend

\begingroup\fontsize{10}{12}\selectfont
\begingroup\fontsize{10}{12}\selectfont

\tagstructbegin{tag=Table}\tagmcbegin{tag=Table}
\begin{longtable}[t]{l>{\raggedright\arraybackslash}p{2cm}}
\caption{\label{tab:ref-points}Summary of the rebuilding reference points.}\\
\toprule
Quantity & 2021 Assessment Values\\
\midrule
\endfirsthead
\caption[]{\label{tab:ref-points}Summary of the rebuilding reference points. \textit{(continued)}}\\
\toprule
Quantity & 2021 Assessment Values\\
\midrule
\endhead

\endfoot
\bottomrule
\endlastfoot
SB0 (millions of eggs) & 55.08\\
SB40 (millions of eggs) & 22.035\\
SB2021 (millions of eggs) & 7.745\\
Year rebuilding begins & 2023\\
Current year & 2021\\
Tmin & 2040\\
Mean generation time (years) & 27\\
Tmax & 2067\\
Pmax & TBD\\
Ttarget & TBD\\
SPRtarget & TBD\\*
\end{longtable}
\leavevmode\tagmcend\tagstructend\par
\endgroup{}
\endgroup{}

\clearpage

\tagstructbegin{tag=H2}\tagmcbegin{tag=H2}

\hypertarget{summary-of-alternatives}{%
\subsection{Summary of alternatives}\label{summary-of-alternatives}}

\leavevmode\tagmcend\tagstructend

\begingroup\fontsize{10}{12}\selectfont

\begin{landscape}\begingroup\fontsize{10}{12}\selectfont

\tagstructbegin{tag=Table}\tagmcbegin{tag=Table}
\begin{longtable}[t]{l>{\raggedright\arraybackslash}p{1.1cm}>{\raggedright\arraybackslash}p{1.1cm}>{\raggedright\arraybackslash}p{1.1cm}>{\raggedright\arraybackslash}p{1.1cm}>{\raggedright\arraybackslash}p{1.1cm}>{\raggedright\arraybackslash}p{1.1cm}>{\raggedright\arraybackslash}p{1.1cm}>{\raggedright\arraybackslash}p{1.1cm}>{\raggedright\arraybackslash}p{1.1cm}}
\caption{\label{tab:reb-options}Results of rebuilding alternatives based on alternative SPR targets for 50 percent probability of recovery based on the assumed removals for 2021-22.}\\
\toprule
 & SPR= .500       & SPR= .600       & SPR= .700       & SPR= .800       & SPR= .900       & Yr=Tmid         & F=0             & 40-10 rule      & ABC Rule       \\
\midrule
\endfirsthead
\caption[]{\label{tab:reb-options}Results of rebuilding alternatives based on alternative SPR targets for 50 percent probability of recovery based on the assumed removals for 2021-22. \textit{(continued)}}\\
\toprule
 & SPR= .500       & SPR= .600       & SPR= .700       & SPR= .800       & SPR= .900       & Yr=Tmid         & F=0             & 40-10 rule      & ABC Rule       \\
\midrule
\endhead

\endfoot
\bottomrule
\endlastfoot
2021 Assumed Removals (mt) & 13.5 & 13.5 & 13.5 & 13.5 & 13.5 & 13.5 & 13.5 & 13.5 & 13.5\\
2022 Assumed Removals (mt) & 13.5 & 13.5 & 13.5 & 13.5 & 13.5 & 13.5 & 13.5 & 13.5 & 13.5\\
2023 ACL (mt) & 1.79 & 1.42 & 0.94 & 0.56 & 0.25 & 1.79 & 0 & 0.04 & 1.79\\
2024 ACL (mt) & 1.95 & 1.57 & 1.05 & 0.63 & 0.29 & 1.95 & 0 & 0.33 & 1.95\\
SPR & 0.537 & 0.6 & 0.7 & 0.8 & 0.9 & 0.537 & 1 & 0.984 & 0.537\\
Ttarget & 2054 & 2051 & 2046 & 2043 & 2042 & 2054 & 2040 & 2050 & 2054\\
Tmax & 2067 & 2067 & 2067 & 2067 & 2067 & 2067 & 2067 & 2067 & 2067\\
Probability of recovery by Tmax & 0.894 & 0.959 & 0.997 & 1 & 1 & 0.894 & 1 & 0.943 & 0.894\\*
\end{longtable}
\leavevmode\tagmcend\tagstructend\par
\endgroup{}
\end{landscape}
\endgroup{}

\tagstructbegin{tag=P}\tagmcbegin{tag=P}

QUESTION: Should ACL actually be ACL contribution?

\leavevmode\tagmcend\tagstructend\par

\clearpage

\tagstructbegin{tag=H2}\tagmcbegin{tag=H2}

\hypertarget{rebuilding-alternative-time-series}{%
\subsection{Rebuilding alternative time series}\label{rebuilding-alternative-time-series}}

\leavevmode\tagmcend\tagstructend

\tagstructbegin{tag=P}\tagmcbegin{tag=P}

TO DO: Correct the captions since these are not just SPR alternatives

\leavevmode\tagmcend\tagstructend\par

\begingroup\fontsize{10}{12}\selectfont
\begingroup\fontsize{10}{12}\selectfont

\tagstructbegin{tag=Table}\tagmcbegin{tag=Table}
\begin{longtable}[t]{l>{\raggedright\arraybackslash}p{1.1cm}>{\raggedright\arraybackslash}p{1.1cm}>{\raggedright\arraybackslash}p{1.1cm}>{\raggedright\arraybackslash}p{1.1cm}>{\raggedright\arraybackslash}p{1.1cm}>{\raggedright\arraybackslash}p{1.1cm}>{\raggedright\arraybackslash}p{1.1cm}>{\raggedright\arraybackslash}p{1.1cm}>{\raggedright\arraybackslash}p{1.1cm}}
\caption{\label{tab:prob-mat}Probability of recovery by year for rebuilding SPR alternatives assuming removals of 13.5 mt in 2021 and 2022, respectively.}\\
\toprule
Year & SPR= .500       & SPR= .600       & SPR= .700       & SPR= .800       & SPR= .900       & Yr=Tmid         & F=0             & 40-10 rule      & ABC Rule       \\
\midrule
\endfirsthead
\caption[]{\label{tab:prob-mat}Probability of recovery by year for rebuilding SPR alternatives assuming removals of 13.5 mt in 2021 and 2022, respectively. \textit{(continued)}}\\
\toprule
Year & SPR= .500       & SPR= .600       & SPR= .700       & SPR= .800       & SPR= .900       & Yr=Tmid         & F=0             & 40-10 rule      & ABC Rule       \\
\midrule
\endhead

\endfoot
\bottomrule
\endlastfoot
2021 & 0.00 & 0.00 & 0.00 & 0.00 & 0.00 & 0.00 & 0.00 & 0.00 & 0.00\\
2022 & 0.00 & 0.00 & 0.00 & 0.00 & 0.00 & 0.00 & 0.00 & 0.00 & 0.00\\
2023 & 0.00 & 0.00 & 0.00 & 0.00 & 0.00 & 0.00 & 0.00 & 0.00 & 0.00\\
2024 & 0.00 & 0.00 & 0.00 & 0.00 & 0.00 & 0.00 & 0.00 & 0.00 & 0.00\\
2025 & 0.00 & 0.00 & 0.00 & 0.00 & 0.00 & 0.00 & 0.00 & 0.00 & 0.00\\
2026 & 0.00 & 0.00 & 0.00 & 0.00 & 0.00 & 0.00 & 0.00 & 0.00 & 0.00\\
2027 & 0.00 & 0.00 & 0.00 & 0.00 & 0.00 & 0.00 & 0.00 & 0.00 & 0.00\\
2028 & 0.00 & 0.00 & 0.00 & 0.00 & 0.00 & 0.00 & 0.00 & 0.00 & 0.00\\
2029 & 0.00 & 0.00 & 0.00 & 0.00 & 0.00 & 0.00 & 0.00 & 0.00 & 0.00\\
2030 & 0.00 & 0.00 & 0.00 & 0.00 & 0.00 & 0.00 & 0.00 & 0.00 & 0.00\\
2031 & 0.00 & 0.00 & 0.00 & 0.00 & 0.00 & 0.00 & 0.00 & 0.00 & 0.00\\
2032 & 0.00 & 0.00 & 0.00 & 0.00 & 0.00 & 0.00 & 0.00 & 0.00 & 0.00\\
2033 & 0.00 & 0.00 & 0.00 & 0.00 & 0.00 & 0.00 & 0.00 & 0.00 & 0.00\\
2034 & 0.00 & 0.00 & 0.00 & 0.00 & 0.00 & 0.00 & 0.00 & 0.00 & 0.00\\
2035 & 0.00 & 0.00 & 0.00 & 0.00 & 0.01 & 0.00 & 0.01 & 0.00 & 0.00\\
2036 & 0.00 & 0.00 & 0.00 & 0.01 & 0.02 & 0.00 & 0.05 & 0.00 & 0.00\\
2037 & 0.00 & 0.00 & 0.01 & 0.02 & 0.07 & 0.00 & 0.12 & 0.00 & 0.00\\
2038 & 0.00 & 0.00 & 0.02 & 0.07 & 0.13 & 0.00 & 0.24 & 0.01 & 0.00\\
2039 & 0.00 & 0.01 & 0.05 & 0.13 & 0.24 & 0.00 & 0.36 & 0.02 & 0.00\\
2040 & 0.01 & 0.02 & 0.10 & 0.21 & 0.35 & 0.01 & 0.52 & 0.04 & 0.01\\
2041 & 0.02 & 0.05 & 0.14 & 0.30 & 0.47 & 0.02 & 0.67 & 0.07 & 0.02\\
2042 & 0.04 & 0.08 & 0.22 & 0.40 & 0.62 & 0.04 & 0.78 & 0.11 & 0.04\\
2043 & 0.06 & 0.11 & 0.28 & 0.51 & 0.72 & 0.06 & 0.88 & 0.15 & 0.06\\
2044 & 0.09 & 0.15 & 0.37 & 0.61 & 0.81 & 0.09 & 0.94 & 0.19 & 0.09\\
2045 & 0.11 & 0.20 & 0.45 & 0.71 & 0.89 & 0.11 & 0.97 & 0.24 & 0.11\\
2046 & 0.16 & 0.25 & 0.53 & 0.78 & 0.94 & 0.16 & 0.98 & 0.30 & 0.16\\
2047 & 0.20 & 0.29 & 0.61 & 0.86 & 0.96 & 0.20 & 0.99 & 0.34 & 0.20\\
2048 & 0.23 & 0.35 & 0.69 & 0.90 & 0.98 & 0.23 & 1.00 & 0.41 & 0.23\\
2049 & 0.28 & 0.42 & 0.75 & 0.94 & 0.98 & 0.28 & 1.00 & 0.46 & 0.28\\
2050 & 0.33 & 0.48 & 0.79 & 0.95 & 1.00 & 0.33 & 1.00 & 0.51 & 0.33\\
2051 & 0.39 & 0.52 & 0.85 & 0.97 & 1.00 & 0.39 & 1.00 & 0.57 & 0.39\\
2052 & 0.44 & 0.58 & 0.89 & 0.98 & 1.00 & 0.44 & 1.00 & 0.60 & 0.44\\
2053 & 0.50 & 0.62 & 0.91 & 0.99 & 1.00 & 0.50 & 1.00 & 0.65 & 0.50\\
2054 & 0.53 & 0.68 & 0.93 & 0.99 & 1.00 & 0.53 & 1.00 & 0.70 & 0.53\\
2055 & 0.57 & 0.72 & 0.95 & 0.99 & 1.00 & 0.57 & 1.00 & 0.73 & 0.57\\
2056 & 0.61 & 0.76 & 0.96 & 1.00 & 1.00 & 0.61 & 1.00 & 0.77 & 0.61\\
2057 & 0.66 & 0.80 & 0.97 & 1.00 & 1.00 & 0.66 & 1.00 & 0.80 & 0.66\\
2058 & 0.70 & 0.83 & 0.98 & 1.00 & 1.00 & 0.70 & 1.00 & 0.83 & 0.70\\
2059 & 0.73 & 0.85 & 0.98 & 1.00 & 1.00 & 0.73 & 1.00 & 0.84 & 0.73\\
2060 & 0.77 & 0.86 & 0.99 & 1.00 & 1.00 & 0.77 & 1.00 & 0.86 & 0.77\\
2061 & 0.79 & 0.88 & 0.99 & 1.00 & 1.00 & 0.79 & 1.00 & 0.87 & 0.79\\
2062 & 0.81 & 0.90 & 0.99 & 1.00 & 1.00 & 0.81 & 1.00 & 0.89 & 0.81\\
2063 & 0.83 & 0.91 & 0.99 & 1.00 & 1.00 & 0.83 & 1.00 & 0.90 & 0.83\\
2064 & 0.85 & 0.93 & 1.00 & 1.00 & 1.00 & 0.85 & 1.00 & 0.92 & 0.85\\
2065 & 0.87 & 0.94 & 1.00 & 1.00 & 1.00 & 0.87 & 1.00 & 0.93 & 0.87\\
2066 & 0.89 & 0.95 & 1.00 & 1.00 & 1.00 & 0.89 & 1.00 & 0.93 & 0.89\\
2067 & 0.89 & 0.96 & 1.00 & 1.00 & 1.00 & 0.89 & 1.00 & 0.94 & 0.89\\
2068 & 0.90 & 0.97 & 1.00 & 1.00 & 1.00 & 0.90 & 1.00 & 0.95 & 0.90\\
2069 & 0.91 & 0.97 & 1.00 & 1.00 & 1.00 & 0.91 & 1.00 & 0.96 & 0.91\\
2070 & 0.92 & 0.98 & 1.00 & 1.00 & 1.00 & 0.92 & 1.00 & 0.97 & 0.92\\
2071 & 0.93 & 0.98 & 1.00 & 1.00 & 1.00 & 0.93 & 1.00 & 0.97 & 0.93\\
2072 & 0.94 & 0.98 & 1.00 & 1.00 & 1.00 & 0.94 & 1.00 & 0.98 & 0.94\\
2073 & 0.94 & 0.99 & 1.00 & 1.00 & 1.00 & 0.94 & 1.00 & 0.98 & 0.94\\
2074 & 0.95 & 0.99 & 1.00 & 1.00 & 1.00 & 0.95 & 1.00 & 0.99 & 0.95\\
2075 & 0.96 & 0.99 & 1.00 & 1.00 & 1.00 & 0.96 & 1.00 & 0.99 & 0.96\\*
\end{longtable}
\leavevmode\tagmcend\tagstructend\par
\endgroup{}
\endgroup{}

\begingroup\fontsize{10}{12}\selectfont
\begingroup\fontsize{10}{12}\selectfont

\tagstructbegin{tag=Table}\tagmcbegin{tag=Table}
\begin{longtable}[t]{l>{\raggedright\arraybackslash}p{1.1cm}>{\raggedright\arraybackslash}p{1.1cm}>{\raggedright\arraybackslash}p{1.1cm}>{\raggedright\arraybackslash}p{1.1cm}>{\raggedright\arraybackslash}p{1.1cm}>{\raggedright\arraybackslash}p{1.1cm}>{\raggedright\arraybackslash}p{1.1cm}>{\raggedright\arraybackslash}p{1.1cm}>{\raggedright\arraybackslash}p{1.1cm}}
\caption{\label{tab:acl-mat}Catches (mt) by year for rebuilding SPR alternatives assuming removals of 13.5 mt in 2021 and 2022, respectively.}\\
\toprule
Year & SPR= .500       & SPR= .600       & SPR= .700       & SPR= .800       & SPR= .900       & Yr=Tmid         & F=0             & 40-10 rule      & ABC Rule       \\
\midrule
\endfirsthead
\caption[]{\label{tab:acl-mat}Catches (mt) by year for rebuilding SPR alternatives assuming removals of 13.5 mt in 2021 and 2022, respectively. \textit{(continued)}}\\
\toprule
Year & SPR= .500       & SPR= .600       & SPR= .700       & SPR= .800       & SPR= .900       & Yr=Tmid         & F=0             & 40-10 rule      & ABC Rule       \\
\midrule
\endhead

\endfoot
\bottomrule
\endlastfoot
2021 & 13.50 & 13.50 & 13.50 & 13.50 & 13.50 & 13.50 & 13.5 & 13.50 & 13.50\\
2022 & 13.50 & 13.50 & 13.50 & 13.50 & 13.50 & 13.50 & 13.5 & 13.50 & 13.50\\
2023 & 1.79 & 1.42 & 0.94 & 0.56 & 0.25 & 1.79 & 0.0 & 0.04 & 1.79\\
2024 & 1.95 & 1.57 & 1.05 & 0.63 & 0.29 & 1.95 & 0.0 & 0.33 & 1.95\\
2025 & 2.13 & 1.74 & 1.17 & 0.71 & 0.32 & 2.13 & 0.0 & 0.67 & 2.13\\
2026 & 2.30 & 1.91 & 1.29 & 0.79 & 0.36 & 2.30 & 0.0 & 1.02 & 2.30\\
2027 & 2.47 & 2.07 & 1.41 & 0.86 & 0.40 & 2.47 & 0.0 & 1.35 & 2.47\\
2028 & 2.62 & 2.23 & 1.53 & 0.94 & 0.44 & 2.62 & 0.0 & 1.64 & 2.62\\
2029 & 2.77 & 2.39 & 1.65 & 1.02 & 0.47 & 2.77 & 0.0 & 1.92 & 2.77\\
2030 & 2.91 & 2.54 & 1.76 & 1.09 & 0.51 & 2.91 & 0.0 & 2.18 & 2.91\\
2031 & 3.04 & 2.69 & 1.88 & 1.17 & 0.55 & 3.04 & 0.0 & 2.41 & 3.04\\
2032 & 3.16 & 2.83 & 1.99 & 1.24 & 0.59 & 3.16 & 0.0 & 2.63 & 3.16\\
2033 & 3.29 & 2.97 & 2.10 & 1.32 & 0.62 & 3.29 & 0.0 & 2.84 & 3.29\\
2034 & 3.41 & 3.12 & 2.21 & 1.40 & 0.66 & 3.41 & 0.0 & 3.04 & 3.41\\
2035 & 3.53 & 3.26 & 2.32 & 1.47 & 0.70 & 3.53 & 0.0 & 3.25 & 3.53\\
2036 & 3.67 & 3.40 & 2.44 & 1.55 & 0.74 & 3.67 & 0.0 & 3.46 & 3.67\\
2037 & 3.82 & 3.55 & 2.55 & 1.63 & 0.78 & 3.82 & 0.0 & 3.67 & 3.82\\
2038 & 3.97 & 3.69 & 2.66 & 1.70 & 0.82 & 3.97 & 0.0 & 3.89 & 3.97\\
2039 & 4.10 & 3.81 & 2.76 & 1.77 & 0.85 & 4.10 & 0.0 & 4.07 & 4.10\\
2040 & 4.23 & 3.93 & 2.86 & 1.84 & 0.89 & 4.23 & 0.0 & 4.25 & 4.23\\
2041 & 4.37 & 4.07 & 2.97 & 1.91 & 0.92 & 4.37 & 0.0 & 4.43 & 4.37\\
2042 & 4.52 & 4.21 & 3.08 & 1.99 & 0.96 & 4.52 & 0.0 & 4.63 & 4.52\\
2043 & 4.65 & 4.33 & 3.18 & 2.06 & 1.00 & 4.65 & 0.0 & 4.79 & 4.65\\
2044 & 4.78 & 4.46 & 3.29 & 2.14 & 1.04 & 4.78 & 0.0 & 4.96 & 4.78\\
2045 & 4.91 & 4.58 & 3.38 & 2.20 & 1.07 & 4.91 & 0.0 & 5.13 & 4.91\\
2046 & 5.05 & 4.71 & 3.49 & 2.28 & 1.11 & 5.05 & 0.0 & 5.29 & 5.05\\
2047 & 5.18 & 4.84 & 3.59 & 2.35 & 1.15 & 5.18 & 0.0 & 5.45 & 5.18\\
2048 & 5.29 & 4.95 & 3.68 & 2.41 & 1.18 & 5.29 & 0.0 & 5.59 & 5.29\\
2049 & 5.40 & 5.05 & 3.77 & 2.48 & 1.22 & 5.40 & 0.0 & 5.72 & 5.40\\
2050 & 5.50 & 5.14 & 3.86 & 2.54 & 1.25 & 5.50 & 0.0 & 5.83 & 5.50\\
2051 & 5.62 & 5.25 & 3.93 & 2.59 & 1.28 & 5.62 & 0.0 & 5.95 & 5.62\\
2052 & 5.73 & 5.37 & 4.02 & 2.66 & 1.31 & 5.73 & 0.0 & 6.05 & 5.73\\
2053 & 5.82 & 5.45 & 4.10 & 2.72 & 1.34 & 5.82 & 0.0 & 6.11 & 5.82\\
2054 & 5.89 & 5.51 & 4.17 & 2.77 & 1.37 & 5.89 & 0.0 & 6.18 & 5.89\\
2055 & 5.96 & 5.59 & 4.23 & 2.81 & 1.39 & 5.96 & 0.0 & 6.23 & 5.96\\
2056 & 6.03 & 5.66 & 4.29 & 2.85 & 1.42 & 6.03 & 0.0 & 6.29 & 6.03\\
2057 & 6.11 & 5.73 & 4.34 & 2.90 & 1.44 & 6.11 & 0.0 & 6.37 & 6.11\\
2058 & 6.19 & 5.80 & 4.41 & 2.94 & 1.46 & 6.19 & 0.0 & 6.43 & 6.19\\
2059 & 6.27 & 5.89 & 4.48 & 2.99 & 1.49 & 6.27 & 0.0 & 6.50 & 6.27\\
2060 & 6.34 & 5.96 & 4.54 & 3.04 & 1.51 & 6.34 & 0.0 & 6.56 & 6.34\\
2061 & 6.40 & 6.02 & 4.58 & 3.07 & 1.53 & 6.40 & 0.0 & 6.62 & 6.40\\
2062 & 6.46 & 6.07 & 4.63 & 3.11 & 1.55 & 6.46 & 0.0 & 6.66 & 6.46\\
2063 & 6.51 & 6.12 & 4.68 & 3.15 & 1.57 & 6.51 & 0.0 & 6.68 & 6.51\\
2064 & 6.54 & 6.15 & 4.72 & 3.17 & 1.59 & 6.54 & 0.0 & 6.72 & 6.54\\
2065 & 6.59 & 6.19 & 4.75 & 3.20 & 1.60 & 6.59 & 0.0 & 6.77 & 6.59\\
2066 & 6.62 & 6.23 & 4.80 & 3.23 & 1.62 & 6.62 & 0.0 & 6.80 & 6.62\\
2067 & 6.66 & 6.27 & 4.82 & 3.26 & 1.64 & 6.66 & 0.0 & 6.82 & 6.66\\
2068 & 6.70 & 6.31 & 4.85 & 3.28 & 1.65 & 6.70 & 0.0 & 6.86 & 6.70\\
2069 & 6.73 & 6.34 & 4.88 & 3.31 & 1.66 & 6.73 & 0.0 & 6.87 & 6.73\\
2070 & 6.77 & 6.37 & 4.92 & 3.33 & 1.68 & 6.77 & 0.0 & 6.91 & 6.77\\
2071 & 6.81 & 6.42 & 4.95 & 3.35 & 1.69 & 6.81 & 0.0 & 6.94 & 6.81\\
2072 & 6.87 & 6.47 & 4.99 & 3.38 & 1.70 & 6.87 & 0.0 & 7.00 & 6.87\\
2073 & 6.91 & 6.50 & 5.02 & 3.41 & 1.72 & 6.91 & 0.0 & 7.02 & 6.91\\
2074 & 6.93 & 6.52 & 5.04 & 3.42 & 1.73 & 6.93 & 0.0 & 7.04 & 6.93\\
2075 & 6.97 & 6.57 & 5.08 & 3.45 & 1.74 & 6.97 & 0.0 & 7.07 & 6.97\\*
\end{longtable}
\leavevmode\tagmcend\tagstructend\par
\endgroup{}
\endgroup{}

\begingroup\fontsize{10}{12}\selectfont
\begingroup\fontsize{10}{12}\selectfont

\tagstructbegin{tag=Table}\tagmcbegin{tag=Table}
\begin{longtable}[t]{l>{\raggedright\arraybackslash}p{1.1cm}>{\raggedright\arraybackslash}p{1.1cm}>{\raggedright\arraybackslash}p{1.1cm}>{\raggedright\arraybackslash}p{1.1cm}>{\raggedright\arraybackslash}p{1.1cm}>{\raggedright\arraybackslash}p{1.1cm}>{\raggedright\arraybackslash}p{1.1cm}>{\raggedright\arraybackslash}p{1.1cm}>{\raggedright\arraybackslash}p{1.1cm}}
\caption{\label{tab:ofl-mat}OFLs (mt) by year for rebuilding SPR alternatives assuming removals of 13.5 mt in 2021 and 2022, respectively.}\\
\toprule
Year & SPR= .500       & SPR= .600       & SPR= .700       & SPR= .800       & SPR= .900       & Yr=Tmid         & F=0             & 40-10 rule      & ABC Rule       \\
\midrule
\endfirsthead
\caption[]{\label{tab:ofl-mat}OFLs (mt) by year for rebuilding SPR alternatives assuming removals of 13.5 mt in 2021 and 2022, respectively. \textit{(continued)}}\\
\toprule
Year & SPR= .500       & SPR= .600       & SPR= .700       & SPR= .800       & SPR= .900       & Yr=Tmid         & F=0             & 40-10 rule      & ABC Rule       \\
\midrule
\endhead

\endfoot
\bottomrule
\endlastfoot
2021 & 2.37 & 2.37 & 2.37 & 2.37 & 2.37 & 2.37 & 2.37 & 2.37 & 2.37\\
2022 & 2.07 & 2.07 & 2.07 & 2.07 & 2.07 & 2.07 & 2.07 & 2.07 & 2.07\\
2023 & 1.79 & 1.79 & 1.79 & 1.79 & 1.79 & 1.79 & 1.79 & 1.79 & 1.79\\
2024 & 1.95 & 1.96 & 1.98 & 1.99 & 2.00 & 1.95 & 2.01 & 2.01 & 1.95\\
2025 & 2.13 & 2.15 & 2.19 & 2.21 & 2.24 & 2.13 & 2.25 & 2.24 & 2.13\\
2026 & 2.30 & 2.34 & 2.40 & 2.44 & 2.48 & 2.30 & 2.51 & 2.47 & 2.30\\
2027 & 2.47 & 2.52 & 2.60 & 2.66 & 2.71 & 2.47 & 2.75 & 2.68 & 2.47\\
2028 & 2.62 & 2.69 & 2.79 & 2.87 & 2.93 & 2.62 & 2.99 & 2.87 & 2.62\\
2029 & 2.77 & 2.85 & 2.97 & 3.07 & 3.16 & 2.77 & 3.23 & 3.06 & 2.77\\
2030 & 2.91 & 3.00 & 3.15 & 3.27 & 3.37 & 2.91 & 3.46 & 3.22 & 2.91\\
2031 & 3.04 & 3.15 & 3.32 & 3.46 & 3.58 & 3.04 & 3.69 & 3.38 & 3.04\\
2032 & 3.16 & 3.28 & 3.48 & 3.65 & 3.79 & 3.16 & 3.91 & 3.53 & 3.16\\
2033 & 3.29 & 3.41 & 3.64 & 3.83 & 3.99 & 3.29 & 4.13 & 3.66 & 3.29\\
2034 & 3.41 & 3.55 & 3.81 & 4.02 & 4.20 & 3.41 & 4.36 & 3.81 & 3.41\\
2035 & 3.53 & 3.68 & 3.96 & 4.20 & 4.40 & 3.53 & 4.58 & 3.93 & 3.53\\
2036 & 3.67 & 3.83 & 4.14 & 4.40 & 4.63 & 3.67 & 4.83 & 4.08 & 3.67\\
2037 & 3.82 & 3.99 & 4.34 & 4.63 & 4.88 & 3.82 & 5.10 & 4.25 & 3.82\\
2038 & 3.97 & 4.14 & 4.52 & 4.84 & 5.12 & 3.97 & 5.37 & 4.40 & 3.97\\
2039 & 4.10 & 4.28 & 4.69 & 5.04 & 5.35 & 4.10 & 5.62 & 4.53 & 4.10\\
2040 & 4.23 & 4.42 & 4.86 & 5.23 & 5.57 & 4.23 & 5.86 & 4.67 & 4.23\\
2041 & 4.37 & 4.57 & 5.04 & 5.44 & 5.79 & 4.37 & 6.11 & 4.81 & 4.37\\
2042 & 4.52 & 4.73 & 5.23 & 5.66 & 6.04 & 4.52 & 6.38 & 4.95 & 4.52\\
2043 & 4.65 & 4.87 & 5.39 & 5.87 & 6.27 & 4.65 & 6.64 & 5.07 & 4.65\\
2044 & 4.78 & 5.01 & 5.58 & 6.07 & 6.51 & 4.78 & 6.90 & 5.19 & 4.78\\
2045 & 4.91 & 5.14 & 5.74 & 6.26 & 6.71 & 4.91 & 7.13 & 5.33 & 4.91\\
2046 & 5.05 & 5.30 & 5.92 & 6.48 & 6.97 & 5.05 & 7.40 & 5.44 & 5.05\\
2047 & 5.18 & 5.44 & 6.10 & 6.68 & 7.20 & 5.18 & 7.66 & 5.56 & 5.18\\
2048 & 5.29 & 5.56 & 6.24 & 6.86 & 7.42 & 5.29 & 7.90 & 5.66 & 5.29\\
2049 & 5.40 & 5.68 & 6.40 & 7.04 & 7.63 & 5.40 & 8.15 & 5.75 & 5.40\\
2050 & 5.50 & 5.78 & 6.55 & 7.21 & 7.81 & 5.50 & 8.37 & 5.84 & 5.50\\
2051 & 5.62 & 5.90 & 6.68 & 7.36 & 8.01 & 5.62 & 8.59 & 5.95 & 5.62\\
2052 & 5.73 & 6.03 & 6.83 & 7.56 & 8.21 & 5.73 & 8.80 & 6.05 & 5.73\\
2053 & 5.82 & 6.13 & 6.96 & 7.73 & 8.42 & 5.82 & 9.04 & 6.11 & 5.82\\
2054 & 5.89 & 6.19 & 7.07 & 7.86 & 8.58 & 5.89 & 9.24 & 6.18 & 5.89\\
2055 & 5.96 & 6.28 & 7.18 & 8.00 & 8.74 & 5.96 & 9.40 & 6.23 & 5.96\\
2056 & 6.03 & 6.36 & 7.28 & 8.11 & 8.88 & 6.03 & 9.59 & 6.29 & 6.03\\
2057 & 6.11 & 6.44 & 7.37 & 8.23 & 9.03 & 6.11 & 9.75 & 6.37 & 6.11\\
2058 & 6.19 & 6.52 & 7.49 & 8.37 & 9.17 & 6.19 & 9.92 & 6.43 & 6.19\\
2059 & 6.27 & 6.62 & 7.60 & 8.49 & 9.32 & 6.27 & 10.09 & 6.50 & 6.27\\
2060 & 6.34 & 6.69 & 7.70 & 8.63 & 9.48 & 6.34 & 10.26 & 6.56 & 6.34\\
2061 & 6.40 & 6.76 & 7.77 & 8.73 & 9.60 & 6.40 & 10.41 & 6.62 & 6.40\\
2062 & 6.46 & 6.82 & 7.86 & 8.83 & 9.73 & 6.46 & 10.58 & 6.66 & 6.46\\
2063 & 6.51 & 6.88 & 7.94 & 8.94 & 9.86 & 6.51 & 10.72 & 6.68 & 6.51\\
2064 & 6.54 & 6.91 & 8.01 & 9.02 & 9.95 & 6.54 & 10.83 & 6.72 & 6.54\\
2065 & 6.59 & 6.96 & 8.06 & 9.10 & 10.06 & 6.59 & 10.96 & 6.77 & 6.59\\
2066 & 6.62 & 7.00 & 8.14 & 9.19 & 10.17 & 6.62 & 11.08 & 6.80 & 6.62\\
2067 & 6.66 & 7.04 & 8.18 & 9.25 & 10.25 & 6.66 & 11.20 & 6.82 & 6.66\\
2068 & 6.70 & 7.09 & 8.24 & 9.32 & 10.36 & 6.70 & 11.31 & 6.86 & 6.70\\
2069 & 6.73 & 7.12 & 8.29 & 9.40 & 10.42 & 6.73 & 11.38 & 6.87 & 6.73\\
2070 & 6.77 & 7.16 & 8.35 & 9.47 & 10.51 & 6.77 & 11.49 & 6.91 & 6.77\\
2071 & 6.81 & 7.21 & 8.40 & 9.53 & 10.59 & 6.81 & 11.60 & 6.94 & 6.81\\
2072 & 6.87 & 7.27 & 8.48 & 9.60 & 10.67 & 6.87 & 11.67 & 7.00 & 6.87\\
2073 & 6.91 & 7.30 & 8.52 & 9.68 & 10.76 & 6.91 & 11.77 & 7.02 & 6.91\\
2074 & 6.93 & 7.33 & 8.55 & 9.72 & 10.82 & 6.93 & 11.85 & 7.04 & 6.93\\
2075 & 6.97 & 7.38 & 8.62 & 9.80 & 10.90 & 6.97 & 11.94 & 7.07 & 6.97\\*
\end{longtable}
\leavevmode\tagmcend\tagstructend\par
\endgroup{}
\endgroup{}

\begingroup\fontsize{10}{12}\selectfont
\begingroup\fontsize{10}{12}\selectfont

\tagstructbegin{tag=Table}\tagmcbegin{tag=Table}
\begin{longtable}[t]{l>{\raggedright\arraybackslash}p{1.1cm}>{\raggedright\arraybackslash}p{1.1cm}>{\raggedright\arraybackslash}p{1.1cm}>{\raggedright\arraybackslash}p{1.1cm}>{\raggedright\arraybackslash}p{1.1cm}>{\raggedright\arraybackslash}p{1.1cm}>{\raggedright\arraybackslash}p{1.1cm}>{\raggedright\arraybackslash}p{1.1cm}>{\raggedright\arraybackslash}p{1.1cm}}
\caption{\label{tab:rel-ssb-mat}Spawning output relative to the 40 percent of unfished spawning output target by year for rebuilding SPR alternatives assuming removals of 13.5 mt in 2021 and 2022, respectively. }\\
\toprule
Year & SPR= .500       & SPR= .600       & SPR= .700       & SPR= .800       & SPR= .900       & Yr=Tmid         & F=0             & 40-10 rule      & ABC Rule       \\
\midrule
\endfirsthead
\caption[]{\label{tab:rel-ssb-mat}Spawning output relative to the 40 percent of unfished spawning output target by year for rebuilding SPR alternatives assuming removals of 13.5 mt in 2021 and 2022, respectively.  \textit{(continued)}}\\
\toprule
Year & SPR= .500       & SPR= .600       & SPR= .700       & SPR= .800       & SPR= .900       & Yr=Tmid         & F=0             & 40-10 rule      & ABC Rule       \\
\midrule
\endhead

\endfoot
\bottomrule
\endlastfoot
2021 & 0.35 & 0.35 & 0.35 & 0.35 & 0.35 & 0.35 & 0.35 & 0.35 & 0.35\\
2022 & 0.30 & 0.30 & 0.30 & 0.30 & 0.30 & 0.30 & 0.30 & 0.30 & 0.30\\
2023 & 0.25 & 0.25 & 0.25 & 0.25 & 0.25 & 0.25 & 0.25 & 0.25 & 0.25\\
2024 & 0.28 & 0.28 & 0.28 & 0.28 & 0.28 & 0.28 & 0.29 & 0.29 & 0.28\\
2025 & 0.30 & 0.31 & 0.31 & 0.32 & 0.32 & 0.30 & 0.32 & 0.32 & 0.30\\
2026 & 0.34 & 0.34 & 0.35 & 0.36 & 0.36 & 0.34 & 0.37 & 0.36 & 0.34\\
2027 & 0.37 & 0.38 & 0.39 & 0.40 & 0.41 & 0.37 & 0.41 & 0.40 & 0.37\\
2028 & 0.40 & 0.41 & 0.42 & 0.44 & 0.45 & 0.40 & 0.46 & 0.44 & 0.40\\
2029 & 0.42 & 0.44 & 0.46 & 0.47 & 0.49 & 0.42 & 0.50 & 0.47 & 0.42\\
2030 & 0.45 & 0.47 & 0.49 & 0.51 & 0.53 & 0.45 & 0.54 & 0.50 & 0.45\\
2031 & 0.48 & 0.50 & 0.53 & 0.55 & 0.57 & 0.48 & 0.59 & 0.54 & 0.48\\
2032 & 0.51 & 0.53 & 0.56 & 0.59 & 0.61 & 0.51 & 0.63 & 0.57 & 0.51\\
2033 & 0.53 & 0.55 & 0.59 & 0.63 & 0.65 & 0.53 & 0.68 & 0.60 & 0.53\\
2034 & 0.56 & 0.58 & 0.63 & 0.66 & 0.70 & 0.56 & 0.72 & 0.63 & 0.56\\
2035 & 0.59 & 0.61 & 0.66 & 0.70 & 0.74 & 0.59 & 0.77 & 0.66 & 0.59\\
2036 & 0.61 & 0.64 & 0.70 & 0.74 & 0.78 & 0.61 & 0.82 & 0.68 & 0.61\\
2037 & 0.64 & 0.67 & 0.73 & 0.78 & 0.82 & 0.64 & 0.86 & 0.71 & 0.64\\
2038 & 0.66 & 0.70 & 0.76 & 0.82 & 0.87 & 0.66 & 0.91 & 0.74 & 0.66\\
2039 & 0.69 & 0.72 & 0.79 & 0.86 & 0.91 & 0.69 & 0.96 & 0.77 & 0.69\\
2040 & 0.71 & 0.75 & 0.82 & 0.89 & 0.95 & 0.71 & 1.01 & 0.79 & 0.71\\
2041 & 0.73 & 0.77 & 0.85 & 0.93 & 0.99 & 0.73 & 1.05 & 0.81 & 0.73\\
2042 & 0.76 & 0.80 & 0.89 & 0.96 & 1.03 & 0.76 & 1.09 & 0.84 & 0.76\\
2043 & 0.78 & 0.82 & 0.92 & 1.00 & 1.08 & 0.78 & 1.14 & 0.86 & 0.78\\
2044 & 0.81 & 0.85 & 0.95 & 1.04 & 1.12 & 0.81 & 1.19 & 0.88 & 0.81\\
2045 & 0.83 & 0.87 & 0.98 & 1.08 & 1.16 & 0.83 & 1.24 & 0.90 & 0.83\\
2046 & 0.85 & 0.90 & 1.01 & 1.11 & 1.20 & 0.85 & 1.28 & 0.93 & 0.85\\
2047 & 0.88 & 0.92 & 1.04 & 1.15 & 1.24 & 0.88 & 1.33 & 0.95 & 0.88\\
2048 & 0.90 & 0.95 & 1.07 & 1.18 & 1.28 & 0.90 & 1.37 & 0.97 & 0.90\\
2049 & 0.92 & 0.97 & 1.10 & 1.22 & 1.32 & 0.92 & 1.42 & 0.98 & 0.92\\
2050 & 0.94 & 0.99 & 1.13 & 1.25 & 1.36 & 0.94 & 1.46 & 1.00 & 0.94\\
2051 & 0.96 & 1.01 & 1.15 & 1.28 & 1.40 & 0.96 & 1.50 & 1.02 & 0.96\\
2052 & 0.98 & 1.03 & 1.17 & 1.31 & 1.43 & 0.98 & 1.54 & 1.03 & 0.98\\
2053 & 1.00 & 1.05 & 1.20 & 1.34 & 1.47 & 1.00 & 1.58 & 1.05 & 1.00\\
2054 & 1.01 & 1.07 & 1.23 & 1.37 & 1.50 & 1.01 & 1.62 & 1.06 & 1.01\\
2055 & 1.02 & 1.08 & 1.25 & 1.39 & 1.53 & 1.02 & 1.66 & 1.07 & 1.02\\
2056 & 1.04 & 1.09 & 1.26 & 1.42 & 1.56 & 1.04 & 1.69 & 1.08 & 1.04\\
2057 & 1.05 & 1.11 & 1.28 & 1.44 & 1.59 & 1.05 & 1.72 & 1.09 & 1.05\\
2058 & 1.06 & 1.12 & 1.30 & 1.46 & 1.61 & 1.06 & 1.75 & 1.11 & 1.06\\
2059 & 1.08 & 1.14 & 1.32 & 1.48 & 1.64 & 1.08 & 1.78 & 1.12 & 1.08\\
2060 & 1.09 & 1.15 & 1.34 & 1.51 & 1.66 & 1.09 & 1.81 & 1.13 & 1.09\\
2061 & 1.10 & 1.17 & 1.36 & 1.53 & 1.69 & 1.10 & 1.84 & 1.14 & 1.10\\
2062 & 1.11 & 1.18 & 1.37 & 1.55 & 1.71 & 1.11 & 1.87 & 1.15 & 1.11\\
2063 & 1.12 & 1.19 & 1.39 & 1.57 & 1.74 & 1.12 & 1.90 & 1.16 & 1.12\\
2064 & 1.13 & 1.20 & 1.40 & 1.58 & 1.76 & 1.13 & 1.92 & 1.16 & 1.13\\
2065 & 1.14 & 1.20 & 1.41 & 1.60 & 1.78 & 1.14 & 1.94 & 1.17 & 1.14\\
2066 & 1.15 & 1.21 & 1.42 & 1.62 & 1.80 & 1.15 & 1.97 & 1.18 & 1.15\\
2067 & 1.15 & 1.22 & 1.43 & 1.63 & 1.81 & 1.15 & 1.99 & 1.18 & 1.15\\
2068 & 1.16 & 1.23 & 1.44 & 1.64 & 1.83 & 1.16 & 2.01 & 1.19 & 1.16\\
2069 & 1.16 & 1.23 & 1.45 & 1.65 & 1.85 & 1.16 & 2.03 & 1.19 & 1.16\\
2070 & 1.17 & 1.24 & 1.46 & 1.67 & 1.86 & 1.17 & 2.04 & 1.19 & 1.17\\
2071 & 1.18 & 1.25 & 1.47 & 1.68 & 1.88 & 1.18 & 2.06 & 1.20 & 1.18\\
2072 & 1.19 & 1.26 & 1.48 & 1.69 & 1.89 & 1.19 & 2.08 & 1.21 & 1.19\\
2073 & 1.19 & 1.27 & 1.49 & 1.70 & 1.90 & 1.19 & 2.09 & 1.22 & 1.19\\
2074 & 1.20 & 1.27 & 1.50 & 1.72 & 1.92 & 1.20 & 2.11 & 1.22 & 1.20\\
2075 & 1.20 & 1.28 & 1.51 & 1.72 & 1.93 & 1.20 & 2.13 & 1.22 & 1.20\\*
\end{longtable}
\leavevmode\tagmcend\tagstructend\par
\endgroup{}
\endgroup{}

\clearpage

\clearpage

\tagstructbegin{tag=H1}\tagmcbegin{tag=H1}

\hypertarget{figures}{%
\section{Figures}\label{figures}}

\leavevmode\tagmcend\tagstructend

\tagstructbegin{tag=Figure,alttext={Probability of rebuilding by year for alternative rebuilding strategies.}}\tagmcbegin{tag=Figure}

\begin{figure}
\centering
\includegraphics[width=1\textwidth,height=1\textheight]{C:/Users/Brian.Langseth/Desktop/ca/rebuilder/930_F2017_2019_ageStruc2021/rebuilding_probability.png}
\caption{Probability of rebuilding by year for alternative rebuilding strategies.\label{fig:prob-fig}}
\end{figure}

\tagmcend\tagstructend

\tagstructbegin{tag=Figure,alttext={Catches (mt) by year for alternative rebuilding strategies.}}\tagmcbegin{tag=Figure}

\begin{figure}
\centering
\includegraphics[width=1\textwidth,height=1\textheight]{C:/Users/Brian.Langseth/Desktop/ca/rebuilder/930_F2017_2019_ageStruc2021/rebuilding_acl.png}
\caption{Catches (mt) by year for alternative rebuilding strategies.\label{fig:acl-fig}}
\end{figure}

\tagmcend\tagstructend

\tagstructbegin{tag=Figure,alttext={Spawning output relative to the management target of 40 percent of unfished spawning output by year for alternative rebuilding strategies.}}\tagmcbegin{tag=Figure}

\begin{figure}
\centering
\includegraphics[width=1\textwidth,height=1\textheight]{C:/Users/Brian.Langseth/Desktop/ca/rebuilder/930_F2017_2019_ageStruc2021/rebuilding_relative_sb.png}
\caption{Spawning output relative to the management target of 40 percent of unfished spawning output by year for alternative rebuilding strategies.\label{fig:rel-ssb-fig}}
\end{figure}

\tagmcend\tagstructend

\tagstructbegin{tag=Figure,alttext={Spawning output by year for alternative rebuilding strategies.}}\tagmcbegin{tag=Figure}

\begin{figure}
\centering
\includegraphics[width=1\textwidth,height=1\textheight]{C:/Users/Brian.Langseth/Desktop/ca/rebuilder/930_F2017_2019_ageStruc2021/rebuilding_ssb.png}
\caption{Spawning output by year for alternative rebuilding strategies.\label{fig:ssb-fig}}
\end{figure}

\tagmcend\tagstructend

\tagstructbegin{tag=P}\tagmcbegin{tag=P}

--\textgreater{}

\leavevmode\tagmcend\tagstructend\par

\tagstructbegin{tag=P}\tagmcbegin{tag=P}

--\textgreater{}

\leavevmode\tagmcend\tagstructend\par

\clearpage

\tagstructbegin{tag=H1}\tagmcbegin{tag=H1}

\hypertarget{appendix}{%
\section{Appendix}\label{appendix}}

\leavevmode\tagmcend\tagstructend

\tagstructbegin{tag=H2}\tagmcbegin{tag=H2}

\hypertarget{append_a}{%
\subsection{Appendix A: Rebuilder data file.}\label{append_a}}

\leavevmode\tagmcend\tagstructend

\tagstructbegin{tag=P}\tagmcbegin{tag=P}

The rebuild.dat file used for the SPR alternative analysis has been provided as a separate file.

\leavevmode\tagmcend\tagstructend\par

\tagstructbegin{tag=H2}\tagmcbegin{tag=H2}

\hypertarget{append_b}{%
\subsection{Appendix B: Biological information used for the sensitivity rebuilder analysis with dome-shaped selectivity and blocks.}\label{append_b}}

\leavevmode\tagmcend\tagstructend

\tagstructbegin{tag=H2}\tagmcbegin{tag=H2}

\hypertarget{append_b}{%
\subsection{Appendix C: Comparison between the approved stock synthesis base case model and the stock synthesis sensitivity model with dome-shaped selectivity and blocks.}\label{append_b}}

\leavevmode\tagmcend\tagstructend

\tagstructbegin{tag=Figure,alttext={Change in estimated spawning output for the adopted stock synthesis base model (Base) and the stock synthesis sensitivity model (RecCom Block 2001). Shading indicates 95 percent confidence intervals around each scenario.}}\tagmcbegin{tag=Figure}

\begin{figure}
\centering
\includegraphics[width=1\textwidth,height=1\textheight]{C:/Users/Brian.Langseth/Desktop/ca/rebuilder/write_up/figures/rebuilder_sensitivities_compare2_spawnbio_uncertainty.png}
\caption{Change in estimated spawning output for the adopted stock synthesis base model (Base) and the stock synthesis sensitivity model (RecCom Block 2001). Shading indicates 95 percent confidence intervals around each scenario.\label{fig:ssb-sens}}
\end{figure}

\tagmcend\tagstructend

\tagstructbegin{tag=Figure,alttext={Change in estimated fraction of unfished for the adopted stock synthesis base model (base) and the stock synthesis sensitivity model that informs the rebuilding sensitivity (RecCom Block 2001). Shading indicates 95 percent confidence intervals around each scenario.}}\tagmcbegin{tag=Figure}

\begin{figure}
\centering
\includegraphics[width=1\textwidth,height=1\textheight]{C:/Users/Brian.Langseth/Desktop/ca/rebuilder/write_up/figures/rebuilder_sensitivities_compare4_Bratio_uncertainty.png}
\caption{Change in estimated fraction of unfished for the adopted stock synthesis base model (base) and the stock synthesis sensitivity model that informs the rebuilding sensitivity (RecCom Block 2001). Shading indicates 95 percent confidence intervals around each scenario.\label{fig:depl-sens}}
\end{figure}

\tagmcend\tagstructend

\tagstructbegin{tag=Figure,alttext={Recreational and commercial selectivity for the stock synthesis sensitvity model with blocks in 2001 and allowing dome-shaped selectivity (red lines) compared to the sensitivity for the adopted base model (blue lines).}}\tagmcbegin{tag=Figure}

\begin{figure}
\centering
\includegraphics[width=1\textwidth,height=1\textheight]{C:/Users/Brian.Langseth/Desktop/ca/rebuilder/write_up/figures/selex_compare.png}
\caption{Recreational and commercial selectivity for the stock synthesis sensitvity model with blocks in 2001 and allowing dome-shaped selectivity (red lines) compared to the sensitivity for the adopted base model (blue lines).\label{fig:selex-sens}}
\end{figure}

\tagmcend\tagstructend

\tagstructbegin{tag=Figure,alttext={Mean length and fit (blue line) for the commercial length data from the adoped stock synthesis base model with 95 percent confidence intervals (solid black bar) based on current samples sizes. Thinner intervals (with capped ends) show result of further adjusting sample sizes based on suggested multiplier for Francis data weighting.}}\tagmcbegin{tag=Figure}

\begin{figure}
\centering
\includegraphics[width=0.9\textwidth,height=0.9\textheight]{C:/Users/Brian.Langseth/Desktop/ca/9_0_0_postSSC_base/plots/comp_lenfit_data_weighting_TA1.8_CA_Commercial.png}
\caption{Mean length and fit (blue line) for the commercial length data from the adoped stock synthesis base model with 95 percent confidence intervals (solid black bar) based on current samples sizes. Thinner intervals (with capped ends) show result of further adjusting sample sizes based on suggested multiplier for Francis data weighting.\label{fig:MeanLengthFit-base}}
\end{figure}

\tagmcend\tagstructend

\tagstructbegin{tag=Figure,alttext={Mean length and fit (blue line) for the commercial length data from the stock synthesis sensitivity model with 95 percent confidence intervals (solid black bar) based on current samples sizes. Thinner intervals (with capped ends) show result of further adjusting sample sizes based on suggested multiplier for Francis data weighting.}}\tagmcbegin{tag=Figure}

\begin{figure}
\centering
\includegraphics[width=0.9\textwidth,height=0.9\textheight]{C:/Users/Brian.Langseth/Desktop/ca/rebuilder/9_4_0_recComBlock2001/plots/comp_lenfit_data_weighting_TA1.8_CA_Commercial.png}
\caption{Mean length and fit (blue line) for the commercial length data from the stock synthesis sensitivity model with 95 percent confidence intervals (solid black bar) based on current samples sizes. Thinner intervals (with capped ends) show result of further adjusting sample sizes based on suggested multiplier for Francis data weighting.\label{fig:MeanLengthFit-sens}}
\end{figure}

\tagmcend\tagstructend

\tagstructbegin{tag=Figure,alttext={Aggregated length comps over all years for the adopted stock synthesis base model.}}\tagmcbegin{tag=Figure}

\begin{figure}
\centering
\includegraphics[width=1\textwidth,height=1\textheight]{C:/Users/Brian.Langseth/Desktop/ca/9_0_0_postSSC_base/plots/comp_lenfit__aggregated_across_time.png}
\caption{Aggregated length comps over all years for the adopted stock synthesis base model.\label{fig:aggLenFit-base}}
\end{figure}

\tagmcend\tagstructend

\tagstructbegin{tag=Figure,alttext={Aggregated length comps over all years for the stock synthesis sensitivity model.}}\tagmcbegin{tag=Figure}

\begin{figure}
\centering
\includegraphics[width=1\textwidth,height=1\textheight]{C:/Users/Brian.Langseth/Desktop/ca/rebuilder/9_4_0_recComBlock2001/plots/comp_lenfit__aggregated_across_time.png}
\caption{Aggregated length comps over all years for the stock synthesis sensitivity model.\label{fig:aggLenFit-sens}}
\end{figure}

\tagmcend\tagstructend

\tagstructbegin{tag=Figure,alttext={Pearson residuals for the recreational and commercial fleets for the adopted stock synthesis base model. Closed bubbles are positive residuals (observed > expected) and open bubbles are negative residuals (observed < expected).}}\tagmcbegin{tag=Figure}

\begin{figure}
\centering
\includegraphics[width=0.9\textwidth,height=0.9\textheight]{C:/Users/Brian.Langseth/Desktop/ca/9_0_0_postSSC_base/plots/comp_lenfit__multi-fleet_comparison.png}
\caption{Pearson residuals for the recreational and commercial fleets for the adopted stock synthesis base model. Closed bubbles are positive residuals (observed \textgreater{} expected) and open bubbles are negative residuals (observed \textless{} expected).\label{fig:Pearson-base}}
\end{figure}

\tagmcend\tagstructend

\tagstructbegin{tag=Figure,alttext={Pearson residuals for the recreational and commercial fleets for the stock synthesis sensitivity model. Closed bubbles are positive residuals (observed > expected) and open bubbles are negative residuals (observed < expected).}}\tagmcbegin{tag=Figure}

\begin{figure}
\centering
\includegraphics[width=0.9\textwidth,height=0.9\textheight]{C:/Users/Brian.Langseth/Desktop/ca/rebuilder/9_4_0_recComBlock2001/plots/comp_lenfit__multi-fleet_comparison.png}
\caption{Pearson residuals for the recreational and commercial fleets for the stock synthesis sensitivity model. Closed bubbles are positive residuals (observed \textgreater{} expected) and open bubbles are negative residuals (observed \textless{} expected).\label{fig:Pearson-sens}}
\end{figure}

\tagmcend\tagstructend
\end{document}
